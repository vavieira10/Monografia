A qualidade de imagens de íris em sistemas de reconhecimento de íris influencia diretamente na etapa de segmentação, assim como a qualidade da segmentação afeta a extração dos atributos que identificam unicamente uma pessoa.  Este projeto propõe, então, o estudo da aplicação de duas métricas de qualidade, uma para avaliar a qualidade de imagens de íris de comprimento de \textit{luz visível} e outra para avaliar a qualidade da etapa de segmentação de íris, em sistemas de reconhecimento de íris. Uma arquitetura de sistemas de reconhecimento de íris é proposta com as métricas \textit{\acrfull{DSMI}} e \textit{\acrfull{FCE}}. Os experimentos são realizados utilizando um sistema de reconhecimento de íris \textit{open source} e quatro bancos de imagens de íris de \textit{luz visível}. A arquitetura proposta melhorou consideravelmente o desempenho de sistemas de reconhecimento de íris nos dois bancos com imagens mais comportadas e melhorou pouco ou piorou o desempenho dos outros dois bancos. No entanto, os dois resultados adversos estão mais relacionados com o banco de imagens do que com a arquitetura proposta, de forma que pode-se afirmar que a arquitetura tem o potencial de ser utilizada em sistemas de reconhecimento de íris.
