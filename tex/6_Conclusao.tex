
\par Neste projeto, foi proposto uma arquitetura de sistema de reconhecimento de íris \textit{\acrshort{LV}} flexível para qualquer algoritmo de segmentação, codificação e correspondência de íris, que utiliza duas métricas de qualidade: uma para avaliar a qualidade de imagens de íris chamada \textit{\acrshort{DSMI}}, e outra para avaliar a qualidade da etapa de segmentação de íris chamada \textit{\acrshort{FCE}}. 

\par Foi sugerido que a utilização de duas métricas de qualidade, diferentemente de usar métricas de qualidade somente antes ou depois da segmentação, melhoraria o desempenho de sistemas de reconhecimento de íris. Para realizar os experimentos, foram utilizados quatro bancos de dados de imagens de íris \textit{\acrshort{LV}}: \textit{MICHE}, \textit{UBIRISv1}, \textit{UBIRISv2} e \textit{\acrfull{Warsaw}}. 

\par Foram realizados quatro experimentos divididos em duas categorias, utilizando as métricas de desempenho \textit{\acrshort{AUC}} e \textit{\acrshort{ROC}}, \textit{\acrshort{EER}} e \textit{\acrshort{d'}}. Apesar dos resultados terem variado para cada banco de dados, onde no \textit{UBIRISv1} e \textit{UBIRISv2} a arquitetura proposta obteve os melhores resultados nas duas métricas de desempenho, \textit{\acrshort{AUC}} e \textit{\acrshort{EER}}, e na métrica \textit{\acrshort{d'}} os resultados obtidos para cada banco de dados dependeram do limiar empregado, pode-se afirmar que utilizar duas métricas de qualidade podem melhorar o desempenho de sistemas de reconhecimento de íris \textit{\acrshort{LV}} e que arquitetura proposta tem potencial para ser aplicada nesses sistemas, mas que estudos mais aprofundados devem ser feitos de forma a generalizar os melhores desempenhos, sem depender do banco de dados.

\section{Trabalhos Futuros} \label{sec:conclusao:trabalhos_futuros}

\par A arquitetura de sistema de reconhecimento de íris de \textit{\acrshort{LV}} proposta apresentou bons resultados, mas só obteve os melhores majoritariamente em dois bancos de dados, enquanto nos outros, somente em alguns casos.

\par Foi observado que a etapa de segmentação foi um dos fatores para essas diferenças de resultados, porque enquanto nos bancos de dados \textit{UBIRISv1} e \textit{\acrshort{Warsaw}} o algoritmo de segmentação do sistema de reconhecimento de íris \textit{OSIRISv4.1} conseguiu segmentar bem as imagens das íris, nos bancos \textit{MICHE} e \textit{UBIRISv2}, a segmentação apresentou resultados ruins, por conta do desafio de segmentar as imagens desses bancos e de limitações do \textit{OSIRISv4.1}. Portanto, outros sistemas de reconhecimento de íris podem ser usados para testar a arquitetura proposta.

\par Apesar do projeto proposto ter consistido em utilizar as métricas de qualidade \textit{\acrshort{DSMI}} e \textit{\acrshort{FCE}}, a arquitetura proposta é flexível para outras métricas de qualidade de imagens de íris e da etapa de segmentação. Portanto, outras métricas de qualidade podem ser aplicadas ou até combinadas com as já utilizadas no sistema para avaliar o desempenho de sistemas de reconhecimento de íris.