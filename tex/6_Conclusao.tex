
\par Neste projeto, foi proposto uma arquitetura de sistema de reconhecimento de íris \textit{\acrshort{LV}} flexível para qualquer algoritmo de segmentação, codificação e correspondência de íris, que utiliza duas métricas de qualidade: uma para avaliar a qualidade de imagens de íris chamada \textit{\acrshort{DSMI}} e outra para avaliar a qualidade da etapa de segmentação de íris chamada \textit{\acrshort{FCE}}. 

\par Foi sugerido no projeto que a utilização de duas métricas de qualidade, diferentemente de usar métricas de qualidade somente antes ou depois da segmentação, melhoraria o desempenho de sistemas de reconhecimento de íris. Para realizar os experimentos, foram utilizados quatro bancos de imagens de íris \textit{\acrshort{LV}}: \textit{MICHE}, \textit{UBIRISv1}, \textit{UBIRISv2} e \textit{\acrfull{Warsaw}}. 

\par Foram realizados quatro experimentos utilizando os índices de desempenho \textit{\acrshort{AUC}} e \textit{\acrshort{ROC}}, \textit{\acrshort{EER}} e \textit{\acrshort{d'}}. Os resultados variaram para cada banco de dados, onde no \textit{UBIRISv1} e \textit{\acrshort{Warsaw}} a arquitetura proposta melhorou substancialmente o desempenho de sistemas de reconhecimento de íris nos índices de desempenho \textit{\acrshort{AUC}} e \textit{\acrshort{EER}}. Já no índice de desempenho \textit{\acrshort{d'}}, os resultados obtidos para cada banco de dados dependeram do limiar empregado. Portanto, utilizar duas métricas de qualidade podem melhorar o desempenho de sistemas de reconhecimento de íris \textit{\acrshort{LV}}. No entanto, estudos mais aprofundados devem ser feitos de forma a generalizar os melhores desempenhos, sem depender do banco de imagens.

\par Foi observado que a etapa de segmentação foi um dos fatores para essas diferenças de resultados, porque enquanto nos bancos de dados \textit{UBIRISv1} e \textit{\acrshort{Warsaw}} o algoritmo de segmentação do sistema de reconhecimento de íris \textit{OSIRISv4.1} conseguiu segmentar corretamente as íris, nos bancos \textit{MICHE} e \textit{UBIRISv2}, as íris, em sua maioria, foram incorretamente segmentadas, por conta do desafio de segmentar as imagens desses bancos e de limitações do \textit{OSIRISv4.1}. Portanto, outros sistemas de reconhecimento de íris com diferentes algoritmos de segmentação podem ser usados para testar a arquitetura proposta.

\par Apesar do projeto proposto ter consistido em utilizar as métricas de qualidade \textit{\acrshort{DSMI}} e \textit{\acrshort{FCE}}, a arquitetura proposta é flexível para outras métricas de qualidade de imagens de íris e da etapa de segmentação. Portanto, outras métricas de qualidade podem ser aplicadas ou até combinadas com aquelas já utilizadas no sistema para avaliar o desempenho de sistemas de reconhecimento de íris.