
\par Neste capítulo, são apresentados os experimentos realizados para analisar a arquitetura proposta utilizando as duas métricas de qualidade: \textit{\acrshort{DSMI}} e \textit{\acrshort{FCE}}. São introduzidos os bancos de dados de imagens de íris \textit{\acrshort{LV}}; os limiares calculados $T_{DSMI}$ e $T_{FCE}$ calculados para os bancos de dados; os ruídos aplicados nas imagens dos bancos de dados para os experimentos; os resultados das segmentações de íris usando o sistema de reconhecimento de íris \textit{OSIRISv4.1}; e, por fim, são apresentados os resultados dos experimentos utilizando três métricas de desempenho de sistemas de reconhecimento de íris: \textit{\acrfull{AUC}} e \textit{ \acrfull{ROC}}, \textit{\acrfull{EER}} e \textit{\acrfull{d'}}.


\section{Bancos de dados} \label{sec:experimentos:db}

\par Para a análise da arquitetura de sistema de reconhecimento de íris proposta com as duas métricas, \textit{\acrshort{DSMI}} e \textit{\acrshort{FCE}}, foram utilizados quatro bancos de dados de íris \textit{\acrshort{LV}}: 

\begin{enumerate}
    \item \textit{MICHE} \cite{marsico2017-MICHE-1, santada2016-MICHE-2, miche};
    \item \textit{UBIRISv1} \cite{proenca2005-ubirisv1, ubirisv1};
    \item \textit{UBIRISv2} \cite{proence2010-ubirisv2, ubirisv2};
    \item \textit{\acrfull{Warsaw}} \cite{trokielwicz2016-Warsaw, warsaw}.
\end{enumerate}

\par De cada banco de dados, foram selecionadas cinco imagens de 40 indivíduos aleatoriamente. Dessas cinco imagens, 2 foram reservadas para treino e 3 para teste. As imagens de treino foram utilizadas para calcular os limiares $T_{DSMI}$, $T_{FCE}$ de cada banco de dados. Já as imagens de teste foram utilizadas para o uso na arquitetura proposta e avaliar como as métricas de qualidade influenciaram no rendimento de sistemas de reconhecimento de íris. Foram selecionadas somente cinco imagens porque entre os bancos de dados, é o número mínimo de imagens por indivíduo.

% \par De cada banco de dados, foram selecionadas cinco imagens de 40 indivíduos aleatoriamente. Dessas cinco imagens, 2 foram reservadas para treino e 3 para teste. As imagens de treino foram utilizadas para calcular os limiares $T_{DSMI}$, $T_{FCE}$ de cada banco de dados e o parâmetro $\beta$ da função de normalização f() (\refEq{eq:fce:normFCM}) para todos os bancos de dados. Já as imagens de teste foram utilizadas para o uso na arquitetura proposta e avaliar como as métricas de qualidade influenciaram no rendimento de sistemas de reconhecimento de íris.

\par As Seções \ref{sec:experimentos:db:miche} até \ref{sec:experimentos:db:warsaw} explicam com mais detalhes e ilustram os bancos de dados enumerados acima e na Seção \ref{sec:experimentos:db:limiares} é explicado como os limiares foram calculados e os resultados obtidos.

\subsection{\textit{MICHE}}\label{sec:experimentos:db:miche}

\par \textit{MICHE} é uma base de dados de imagens de íris capturadas por \textit{smartphones}, cujos objetivos são entregar uma larga quantidade de indivíduos, usar mais de um \textit{smartphone} para capturar as imagens, simular situações reais em que as pessoas tiram as próprias fotos e sessões para a aquisição das imagens em tempos separados \cite{santada2016-MICHE-2}. O banco de dados contém imagens de 75 indivíduos, aonde para cada pessoa, somente imagens de um dos olhos é capturada e pelo menos 4 imagens devem ser capturadas nos \textit{smartphones}:

\begin{itemize}
    \item \textit{Galaxy Samsung IV}: 1297 imagens;
    \item \textit{iPhone5}: 1262 imagens;
    \item \textit{Galaxy Tablet II}: 632 imagens.
\end{itemize}

\par As imagens são capturadas pelos próprios voluntários, aonde eles podem ou não estar de óculos, são capturadas em distâncias diferentes e em dois ambientes diferentes: ao ar livre e em lugares fechados. As imagens do bancos de dados são extremamente desafiadoras para algoritmos de segmentação, porque como os indivíduos da amostra que capturaram as imagens, ruídos como cabelo, fundos na imagem, baixa resolução, problemas de foco, borrão em movimento e distorções de iluminação são inevitáveis.

\par Neste projeto, foram utilizadas somente as imagens capturadas pelo \textit{smartphone} \textit{iPhone5}.

\par A \refFig{fig:experimentos:miche_boas} ilustra imagens de boa qualidade do banco de dados \textit{MICHE} e capturadas em ambientes e distâncias diferentes. Já a \refFig{fig:experimentios:miche_ruim} ilustra imagens ruídosas, com óculos e com fundo.

\begin{figure}[h!]
\begin{subfigure}{.3\textwidth}
\centering
\includegraphics[width=4cm,height=4cm]{img/Resultados/miche/boa_1.jpg}
\end{subfigure}\hfill
\begin{subfigure}{.3\textwidth}
\centering
\includegraphics[width=4cm,height=4cm]{img/Resultados/miche/boa_2.jpg}
\end{subfigure}\hfill
\begin{subfigure}{.3\textwidth}
\centering
\includegraphics[width=4cm,height=4cm]{img/Resultados/miche/cabelo.jpg}
\end{subfigure}
\caption{Imagens com qualidade superior do \textit{MICHE}.}
\label{fig:experimentos:miche_boas}
\end{figure}

\begin{figure}[h!]
\begin{subfigure}{.3\textwidth}
\centering
\includegraphics[width=4cm,height=4cm]{img/Resultados/miche/oculos.jpg}
\end{subfigure}\hfill
\begin{subfigure}{.3\textwidth}
\centering
\includegraphics[width=4cm,height=4cm]{img/Resultados/miche/ruidosa.jpg}
\end{subfigure}\hfill
\begin{subfigure}{.3\textwidth}
\centering
\includegraphics[width=4cm,height=4cm]{img/Resultados/miche/ruidosa_funda.jpg}
\end{subfigure}
\caption{Imagens com qualidade inferior do \textit{MICHE}.}
\label{fig:experimentos:miche_boas}
\end{figure}

\FloatBarrier

\subsection{\textit{UBIRISv1}}\label{sec:experimentos:db:ubirisv1}

\par \textit{UBIRISv1} é um banco de dados de imagens de íris capturadas por câmera fotográfica com o objetivo de capturar imagens com vários ruídos, de forma a simular ambientes não controlados de aquisição \cite{proenca2005-ubirisv1, ubirisv1}. O banco de dados possui imagens de 241 indivíduos, totalizando 1877 imagens. As imagens são capturadas pela câmera \textit{Nikon E5700} e foram tiradas em duas sessões, onde:

\begin{itemize}
    \item Primeira sessão: Imagens foram capturadas com o objetivo de minimizar ruídos, como reflexo, luminosidade e contraste, e a estrutura para capturar as imagens foi montada em uma sala escura;
    \item Segunda sessão: Local de captura foi mudado, para incorporar fatores de luminosidade natural, de forma que ruídos como reflexos, problemas de foco, contraste aparecem nas imagens.
\end{itemize}

\par Neste projeto, foram separadas imagens de 20 indivíduos da primeira sessão e 20 da segunda sessão.

\par As Figuras \ref{fig:experimentos:ubirisv1_sessao1} e \ref{fig:experimentos:ubirisv1_sessao2} ilustram imagens capturadas na primeira e segunda sessão dos mesmos indivíduos, respectivamente.

\begin{figure}[h!]
\begin{subfigure}{.3\textwidth}
\centering
\includegraphics[width=4cm,height=3.5cm]{img/Resultados/ubirisv1/sessao1_1.jpg}
\end{subfigure}\hfill
\begin{subfigure}{.3\textwidth}
\centering
\includegraphics[width=4cm,height=3.5cm]{img/Resultados/ubirisv1/sessao1_55.jpg}
\end{subfigure}\hfill
\begin{subfigure}{.3\textwidth}
\centering
\includegraphics[width=4cm,height=3.5cm]{img/Resultados/ubirisv1/sessao1_76.jpg}
\end{subfigure}
\caption{Imagens da primeira sessão do \textit{UBIRISv1}.}
\label{fig:experimentos:ubirisv1_sessao1}
\end{figure}

\begin{figure}[h!]
\begin{subfigure}{.3\textwidth}
\centering
\includegraphics[width=4cm,height=3.5cm]{img/Resultados/ubirisv1/sessao2_1.jpg}
\end{subfigure}\hfill
\begin{subfigure}{.3\textwidth}
\centering
\includegraphics[width=4cm,height=3.5cm]{img/Resultados/ubirisv1/sessao2_55.jpg}
\end{subfigure}\hfill
\begin{subfigure}{.3\textwidth}
\centering
\includegraphics[width=4cm,height=3.5cm]{img/Resultados/ubirisv1/sessao2_76.jpg}
\end{subfigure}
\caption{Imagens da segunda sessão do \textit{UBIRISv1}.}
\label{fig:experimentos:ubirisv1_sessao2}
\end{figure}

\FloatBarrier

\subsection{\textit{UBIRISv2}}\label{sec:experimentos:db:ubirisv2}

\par \textit{UBIRISv2} é um banco de dados de imagens de íris dos mesmos autores do \textit{UBIRISv1}, com o objetivo de deixar as imagens mais realistas, ou seja, com mais tipos de ruídos e em diversas distâncias \cite{proence2010-ubirisv2}. O banco de dados é composto por 11102 imagens distribuídas em 261 indivíduos. As imagens foram capturadas pela câmera \textit{Canon EOS 5D}.

\par A captura de imagens foi dividida em duas sessões, onde somente a localização da câmera e o tipo de luz artificial no ambiente mudaram.Três imagens são capturadas em cinco distâncias diferentes da câmera, entre 4 e 8 metros, totalizando 15 imagens por indivíduo em cada sessão: 

\begin{itemize}
    \item 8 metros: I1-I3;
    \item 7 metros: I4-I6;
    \item 6 metros: I7-I9;
    \item 5 metros: I10-I12;
    \item 4 metros: I13-I15.
\end{itemize}

\par Foi solicitado aos indivíduos que olhassem para pontos diferentes no ambiente enquanto andavam lentamente entre as marcas das distância, de forma a capturar imagens da íris em ângulos diferentes e em movimento para introduzir ruídos.

\par No projeto, foram utilizadas somente imagens da primeira sessão do banco de dados e imagens capturadas a 4 e 5 metros da câmera (I11-I15), por conta de algumas restrições dos parâmetros de tamanho mínimo e máximo da pupila e íris do algoritmo de segmentação do sistema \textit{OSIRISv4.1}.

\par A \refFig{fig:experimentos:ubirisv2} ilustra exemplos de imagens das cinco distâncias usadas do \textit{UBIRISv2}.

\begin{figure}[htb]
    \centering % <-- added
\begin{subfigure}{0.25\textwidth}
  \includegraphics[width=\linewidth]{img/Resultados/ubirisv2/C3_S1_I11.jpg}
  \caption{I11.}
\end{subfigure}\hfil % <-- added
\begin{subfigure}{0.25\textwidth}
  \includegraphics[width=\linewidth]{img/Resultados/ubirisv2/C3_S1_I12.jpg}
  \caption{I12.}
  \label{fig:2}
\end{subfigure}\hfil % <-- added
\begin{subfigure}{0.25\textwidth}
  \includegraphics[width=\linewidth]{img/Resultados/ubirisv2/C3_S1_I13.jpg}
  \caption{I13.}
\end{subfigure}

\medskip
\begin{subfigure}{0.25\textwidth}
  \includegraphics[width=\linewidth]{img/Resultados/ubirisv2/C3_S1_I14.jpg}
  \caption{I14.}
\end{subfigure}\hfil % <-- added
\begin{subfigure}{0.25\textwidth}
  \includegraphics[width=\linewidth]{img/Resultados/ubirisv2/C3_S1_I15.jpg}
  \caption{I15.}
\end{subfigure}\hfil % <-- added
\caption{Imagens do banco de dados \textit{UBIRISv2} capturadas a 5 e 4 metros da câmera.}
\label{fig:experimentos:ubirisv2}
\end{figure}

\FloatBarrier

\subsection{\textit{\acrshort{Warsaw}}}\label{sec:experimentos:db:warsaw}

\par \textit{\acrshort{Warsaw}} é um banco de dados de imagens de íris capturadas por \textit{smartphones} \cite{trokielwicz2016-Warsaw}. Tem o objetivo de propor imagens de íris de \textit{smartphones} de boa qualidade e avaliar seu rendimento em sistemas de reconhecimento de íris conhecidos. Participaram do projeto 70 voluntários, onde a captura das imagens de suas íris esquerda e direita foi dividida em duas sessões e em ambientes fechados. As imagens foram capturadas pelo \textit{smartphone} \textit{iPhone 5s} e em todas, foi utilizada a câmera traseira com \textit{flash}. No total, foram capturadas 3192 imagens de 139 e 136 íris diferentes dos 70 voluntários.

\par No projeto, foram utilizadas somente imagens da primeira sessão e os olhos era selecionados alternadamente entre os indivíduos.

\par A \refFig{fig:experimentos:warsaw_1} ilustra imagens da primeira sessão do banco de dados \textit{\acrshort{Warsaw}}.

\begin{figure}[H]
    \centering % <-- added
\begin{subfigure}{0.25\textwidth}
  \includegraphics[width=\linewidth]{img/Resultados/warsaw/left_1.jpg}
  \caption{}
\end{subfigure}\hfil % <-- added
\begin{subfigure}{0.25\textwidth}
  \includegraphics[width=\linewidth]{img/Resultados/warsaw/left_17.jpg}
  \caption{}
\end{subfigure}\hfil % <-- added
\begin{subfigure}{0.25\textwidth}
  \includegraphics[width=\linewidth]{img/Resultados/warsaw/left_65.jpg}
  \caption{}
\end{subfigure}

\medskip
\begin{subfigure}{0.25\textwidth}
  \includegraphics[width=\linewidth]{img/Resultados/warsaw/right_22.jpg}
  \caption{}
\end{subfigure}\hfil % <-- added
\begin{subfigure}{0.25\textwidth}
  \includegraphics[width=\linewidth]{img/Resultados/warsaw/right_32.jpg}
  \caption{}
\end{subfigure}\hfil % <-- added
\begin{subfigure}{0.25\textwidth}
  \includegraphics[width=\linewidth]{img/Resultados/warsaw/right_68.jpg}
  \caption{}
\end{subfigure}
\caption{As imagens (a)-(c) ilustram olhos esquerdos e (d)-(f) olhos direitos da primeira sessão do \textit{\acrshort{Warsaw}}.}
\label{fig:experimentos:warsaw_1}
\end{figure}

\FloatBarrier

\subsection{Limiares $T_{DSMI}$ e $T_{FCE}$} \label{sec:experimentos:db:limiares}

\par Como os banco de dados escolhidos para os experimentos possuem características diferentes, os limiares $T_{DSMI}$ e $T_FCE$ da arquitetura proposta (\refFig{fig:metodologia:arquitetura}) foram calculados separadamente para cada banco de dados.

\par Para calcular o limiar $T_{DSMI}$m foram usadas as duas imagens de treino de 40 indivíduos dos bancos de dados, totalizando 80 por banco de dados. As etapas para o cálculo do limiar de cada banco de dados são enumeradas abaixo:

\begin{enumerate}
    \item Calcular a métrica de qualidade \textit{\acrshort{DSMI}} das 80 imagens;
    \item Gerar o histograma das métricas calculadas;
    \item O limiar $T_{DSMI}$ consiste no limiar \textit{Otsu} (\ref{sec:dom_esp:otsu}) calculado a partir do histograma.
\end{enumerate}

\par O cálculo do limiar $T_{FCE}$ seguiu os mesmos passos descritos para o limiar $T_{DSMI}$, mas antes foram executadas as etapas de segmentação do \textit{OSIRISv4.1} das 80 imagens de cada banco de dados e ao invés de calcular a qualidade das imagens de íris usando a métrica \textit{\acrshort{DSMI}}, foram calculadas as qualidades das segmentações pela métrica de qualidade \textit{\acrshort{FCE}}. Os limiares calculados para cada banco de dados são mostrados na \refTab{tab:experimentos:limiares_dsmi_fce}.

\begin{table}[h!]
\centering
\captionof{table}{Valores dos limiares $T_{DSMI}$ e $T_{FCE}$ calculados para todos os bancos de dados.} \label{tab:experimentos:limiares_dsmi_fce} 
\begin{tabular}{lllll}
\cline{2-5}
\multicolumn{1}{l|}{}             & \multicolumn{1}{l|}{\textit{\textbf{MICHE}}} & \multicolumn{1}{l|}{\textit{\textbf{UBIRISv1}}} & \multicolumn{1}{l|}{\textit{\textbf{UBIRISv2}}} & \multicolumn{1}{l|}{\textit{\begin{tabular}[c]{@{}l@{}}\textbf{Warsaw}\end{tabular}}} \\ \hline
\multicolumn{1}{|l|}{\textit{\textbf{$T_{DSMI}$}}} & \multicolumn{1}{l|}{0.75}           & \multicolumn{1}{l|}{0.83}              & \multicolumn{1}{l|}{0.69}              & \multicolumn{1}{l|}{0.51}                                                                                    \\ \hline
\multicolumn{1}{|l|}{\textit{\textbf{$T_{FCE}$}}} & \multicolumn{1}{l|}{0.54}           & \multicolumn{1}{l|}{0.85}              & \multicolumn{1}{l|}{0.54}              & \multicolumn{1}{l|}{0.9}                                                                                     \\ \hline
                                  &                                     &                                        &                                        &                                                                                                 
\end{tabular}
\end{table}

\par O limiar de \textit{Otsu} calcula o limiar que maximiza a variância entre duas classes (Seção \ref{sec:dom_esp:otsu}). No caso, as classes consideradas no cálculos dos limiares $T_{DSMI}$ e $T_{FCE}$ são imagens ou segmentações ruins e boas. Pela tabela, é possível notar como os valores encontrados para os dois limiares variaram para cada banco de dados. O valor do limiar $T_{DSMI}$  do \textit{MICHE} é elevado, porque suas imagens apresentam boas qualidades e o valor do limiar $T_{FCE}$ é baixo, porque como as imagens são capturadas pelas pessoas, o algoritmo de segmentação do \textit{OSIRISv4.1} apresentou resultados ruins. Os limiares do banco de dados \textit{UBIRISv1} apresentaram valores condizentes, onde a qualidade de suas imagens é elevada e a segmentação também. O banco de dados \textit{UBIRISv2} é o mais desafiador entre os escolhidos, porque as imagens apresentam ruídos e causam dificuldades para a segmentação, por conta das diferenças de distância que as imagens foram capturadas, como explicado na Seção \ref{sec:experimentos:db:ubirisv2}, e apresentou valores de limiares esperados para suas condições. O banco de dados \textit{\acrshort{Warsaw}} foi o que apresentou o resultado menos esperado para o limiar $T_{DSMI}$, porque as suas imagens, analisando subjetivamente, são muito boas, como ilustrado nas Figuras \ref{fig:experimentos:warsaw_1} e \ref{fig:experimentos:warsaw_2}, e não apresentaram valores esperados pela métrica de qualidade \textit{\acrshort{DSMI}}, enquanto que o valor do limiar $T_{FCE}$ foi esperado, porque as segmentações apresentaram bons resultados.

\FloatBarrier

\subsection{Ruídos} \label{sec:experimentos:ruidos}

\par Conforme explicado na Seção \ref{sec:experimentos:db}, cinco imagens de 40 indivíduos foram escolhidos aleatoriamente em todos os bancos de dados. Dessas cinco imagens, duas foram reservadas para o cálculo dos limiares $T_{DSMI}$ e $T_{FCE}$ e três para a análise do desempenho da arquitetura de sistema de reconhecimento de íris proposta usando as métricas \textit{\acrshort{DSMI}} e \textit{\acrshort{FCE}}. 

\par Como a qualidade das imagens podem influenciar diretamente na segmentação de íris e três imagens de teste serem muito poucas para os experimentos, por conta da limitação que os bancos de dados imporam, foram geradas imagens com seis ruídos artificiais: \textit{Desfoque Gaussiano}, \textit{Impulso (Sal e Pimenta)}, \textit{\acrfull{WGN}}, \textit{Superexposição}, \textit{Desfoque Móvel} e o padrão de compressão \textit{JPEG2000}. Os cinco primeiros ruídos foram escolhidos porque são os ruídos mais comuns em imagens de íris e são utilizados nos experimentos da métrica \textit{\acrshort{DSMI}} e \cite{Jenadeleh_2018_CVPR_Workshops} e a técnica de compressão \textit{JPEG2000} é a utilizada nos experimentos da métrica \textit{\acrshort{FCE}} \cite{du2010}. Para cada ruído, são usados quatro parâmetros diferentes, de forma que são geradas quatro imagens ruidosas. Os ruídos são gerados para cada imagem de teste, de forma que para cada, 24 novas imagens são geradas e no final, totalizam 3000 imagens para serem usadas nos experimentos em cada banco de dados.

\par Ruídos em imagens digitais são distorções causadas por variações aleatórias no brilho ou cor das imagens, podem ser causados por fatores como o sensor da câmera, iluminação, desfoque, compressão e diminuem a qualidade das imagens \cite{gonsalez2006}. O ruído \textit{Desfoque Gaussiano} consiste em ruídos causados pela função \textit{Gaussiana} e causam suavizações nas imagens \cite{gonsalez2006, boyat2015review}. O ruído de Impulso \textit{Sal e Pimenta} consiste em distúrbios no sinal da imagem que provocam \textit{pixels} brancos e pretos distribuídos na imagem \cite{gonsalez2006, boyat2015review}. O ruído \textit{\acrshort{WGN}} consiste em um ruído causado por sinais aleatórios com intensidades uniformes em frequências que seguem a distribuição \textit{Gaussiana} e causam distorções distribuídas na imagem \cite{boyat2015review}. \textit{Superexposição} é um ruído causado por muita iluminação na hora da captura de imagens e faz com que as intensidades dos \textit{pixels} das imagens tenham valores elevados \cite{overexposure}. \textit{Desfoque Móvel} é o ruído causado por movimento da câmera sendo usada para capturar alguma imagem, no movimento do objeto ou paisagem sendo capturada pela câmera, e causam desfoques na imagem \cite{jiang2005motion}. O padrão de compressão com perdas \textit{JPEG2000} consiste na utilização de técnicas para reduzir o tamanho da imagem, podendo perder a sua qualidade original, e quanto maior a taxa de compressão escolhida, mais distorções são inseridas na imagem resultante \cite{marcellin2000-jpeg2000}. A \refTab{tab:experimentos:ruidos} lista os ruídos descritos acima com os parâmetros utilizados para gerá-los.

\begin{table}[h!]
\centering
\caption{Ruídos gerados e os parâmetros utilizados}
\label{tab:experimentos:ruidos}
\begin{tabular}{|l|l|}
\hline
\multicolumn{1}{|c|}{\textbf{Ruídos}} & \multicolumn{1}{c|}{\textbf{Parâmetros}} \\ \hline
\textit{Desfoque Gaussiano} & {}$\sigma$ = 0.5, 2, 3.5, 5 {} \\ \hline
\textit{Impulso (Sal e Pimenta)} & {}Densidade = 0.05, 0.2, 0.35, 0.5 {} \\ \hline
\textit{\acrshort{WGN}} & {}$\mu = 0$; $\sigma^2$ = 0.002, 0.008, 0.014, 0.02{} \\ \hline
\textit{Superexposição} & {}Constante = 10, 40, 70, 100 {} \\ \hline
\textit{Desfoque Móvel} & {}Comprimento, $\theta$ = 10, 26.66, 43.33, 60{} \\ \hline
\textit{JPEG2000} & {}Taxa de compressão = 25, 50, 75, 100{} \\ \hline
\end{tabular}
\end{table}

\par A \refFig{fig:experimentos:ruidos} ilustra exemplos de imagens de íris com os ruídos gerados.

\begin{figure}[H]
    \centering % <-- added
\begin{subfigure}{0.25\textwidth}
  \includegraphics[width=\linewidth]{img/Resultados/ruidos/gaussian5.jpg}
  \caption{\textit{Desfoque móvel}, $\sigma=5$.}
\end{subfigure}\hfil % <-- added
\begin{subfigure}{0.25\textwidth}
  \includegraphics[width=\linewidth]{img/Resultados/ruidos/impulse0,35.jpg}
  \caption{\textit{Impulso (Sal e Pimenta)}, Densidade = 0.35.}
\end{subfigure}\hfil % <-- added
\begin{subfigure}{0.25\textwidth}
  \includegraphics[width=\linewidth]{img/Resultados/ruidos/wgn0,02.jpg}
  \caption{\textit{\acrshort{WGN}}, $\sigma^2 = 0.02$.}
\end{subfigure}

\medskip
\begin{subfigure}{0.25\textwidth}
  \includegraphics[width=\linewidth]{img/Resultados/ruidos/over_exposure100.jpg}
  \caption{\textit{Superexposição}, Constante = 100.}
\end{subfigure}\hfil % <-- added
\begin{subfigure}{0.25\textwidth}
  \includegraphics[width=\linewidth]{img/Resultados/ruidos/motion_blur60.jpg}
  \caption{\textit{Desfoque móvel}, Comprimento e $\theta$ = 60.}
\end{subfigure}\hfil % <-- added
\begin{subfigure}{0.25\textwidth}
  \includegraphics[width=\linewidth]{img/Resultados/ruidos/jpeg2000_100.jpg}
  \caption{\textit{JPEG2000}, Taxa de compressão = 100.}
\end{subfigure}
\caption{Imagens com os ruídos gerados.}
\label{fig:experimentos:ruidos}
\end{figure}

\FloatBarrier

\section{Segmentação} \label{sec:experimentos:segmentacao}

\par A segmentação é a etapa mais importante de um sistema de reconhecimento de íris, porque da íris segmentada que são extraídos os atributos necessários para sua codificação, de forma que identificam unicamente um indivíduo. Neste projeto, o objetivo é a análise das duas métricas de qualidade, \textit{\acrshort{DSMI}} e \textit{\acrshort{FCE}}, no rendimento de sistemas de reconhecimento de íris e, como explicado na Seção \ref{sec:metodologia:arquitetura}, foi escolhido o sistema de reconhecimento de íris \textit{OSIRISv4.1}. 

\par O algoritmo de segmentação do \textit{OSIRISv4.1} consiste em encontrar primeiro a pupila e depois encontrar os contornos da íris com base em intervalos de valores para os raios, como explicado na Seção \ref{sec:metodologia:segmentacao}. Em seus arquivos de configuração, devem ser passados quatro argumentos: diâmetros mínimos e máximos da pupila e íris. Os parâmetros utilizados no projeto são descritos nas Tabelas \ref{tab:experimentos:diametros} e \ref{tab:experimentos:diametros_ubirisv2}, onde na primeira são ilustrados os parâmetros utilizados nos bancos de dados \textit{MICHE}, \textit{UBIRISv1} e \textit{\acrshort{Warsaw}}, enquanto na segunda os parâmetros utilizados no \textit{UBIRISv2}.

\begin{table}[H]
\centering
\caption{Parâmetros de diâmetros da pupila e íris usados no sistema \textit{OSIRISv4.1} para os bancos de dados \textit{MICHE}, \textit{UBIRISv1} e \textit{Warsaw}.}
\label{tab:experimentos:diametros}
\begin{tabular}{l|l|l|l|}
\cline{2-4}
 & \multicolumn{1}{c|}{\textit{\textbf{MICHE}}} & \multicolumn{1}{c|}{\textit{\textbf{UBIRISv1}}} & \multicolumn{1}{c|}{\textit{\textbf{Warsaw}}} \\ \hline
\multicolumn{1}{|l|}{\textbf{Pupila (Min, Max)}} & (75, 160) & (80, 160) & (50, 200) \\ \hline
\multicolumn{1}{|l|}{\textbf{Íris (Min, Max)}} & (160, 500) & (160, 450) & (100, 400) \\ \hline
\end{tabular}
\end{table}

\begin{table}[H]
\centering
\caption{Parâmetros de diâmetros da pupila e íris usados no sistema \textit{OSIRISv4.1} para o banco de dados \textit{UBIRISv2}.}
\label{tab:experimentos:diametros_ubirisv2}
\begin{tabular}{l|l|l|l|l|}
\cline{2-5}
 & \multicolumn{1}{c|}{\textbf{I11}} & \multicolumn{1}{c|}{\textbf{I12}} & \multicolumn{1}{c|}{\textbf{I13 e I14}} & \multicolumn{1}{c|}{\textbf{I15}} \\ \hline
\multicolumn{1}{|l|}{\textbf{Pupila (Min, Max)}} & (54, 77) & (60, 70) & (60, 70) & (40, 80) \\ \hline
\multicolumn{1}{|l|}{\textbf{Íris (Min, Max)}} & (99, 110) & (100, 120) & (100, 190) & (100, 200) \\ \hline
\end{tabular}
\end{table}

\par O algoritmo de segmentação do \textit{OSIRISv4.1} apresentou algumas limitações no entanto, especialmente para imagens do banco de dados \textit{UBIRISv2}. O \textit{OSIRISv4.1} exige valores mínimos para os parâmetros da pupila e íris, sendo 21 e 99 para os mínimos, e 91 e 399 para os máximos do diâmetro da pupila e íris, respectivamente. Esses valores mínimos e máximos são bons para imagens de íris com dimensões médias e altas, que não é o caso das imagens do \textit{UBIRISv2}, que possuem dimensões de 400x300. Com essas dimensões, as íris e pupilas nas imagens possuem raios pequenos, de forma que provocou erros no algoritmo de segmentação do \textit{OSIRISv4.1} ao tentar usar parâmetros gerais para o banco de dados. A solução encontrada foi, como descrito na análise da Seção \ref{sec:experimentos:db:limiares}, escolher as imagens do \textit{UBIRISv2} capturadas mais próximas da câmera, de forma que a pupila e a íris possuem raios com valores mais elevados (I11-I15), e utilizar parâmetros separados para cada imagem e distância. 

\par Mesmo com as soluções propostas, as segmentações do \textit{UBIRISv2} foram as que apresentaram os piores resultados, principalmente na segmentação da pupila. A \refFig{fig:experimentos:segmentacoes:todos} ilustra resultados bons e ruins do processo de segmentação dos bancos de dados\textit{MICHE, UBIRISv1} e \textit{\acrshort{Warsaw}}, enquanto a \refFig{fig:experimentos:segmentacoes:ubirisv2} ilustra os resultados da segmentação do \textit{UBIRISv2}.

\begin{figure}[H]
    \centering % <-- added
\begin{subfigure}{0.25\textwidth}
  \includegraphics[width=\linewidth]{img/Resultados/miche/miche_seg_boa.jpg}
  \caption{Segmentação boa do banco de dados \textit{MICHE}.}
\end{subfigure}\hfil % <-- added
\begin{subfigure}{0.25\textwidth}
  \includegraphics[width=\linewidth]{img/Resultados/miche/miche_seg_ruim.jpg}
  \caption{Segmentação ruim do banco de dados \textit{MICHE}.}
\end{subfigure}\hfil % <-- added
\begin{subfigure}{0.25\textwidth}
  \includegraphics[width=\linewidth]{img/Resultados/ubirisv1/ubirisv1_seg_boa.jpg}
  \caption{Segmentação boa do banco de dados \textit{UBIRISv1}.}
\end{subfigure}

\medskip
\begin{subfigure}{0.25\textwidth}
  \includegraphics[width=\linewidth]{img/Resultados/ubirisv1/ubirisv1_seg_ruim.jpg}
  \caption{Segmentação ruim do banco de dados \textit{UBIRISv1}.}
\end{subfigure}\hfil % <-- added
\begin{subfigure}{0.25\textwidth}
  \includegraphics[width=\linewidth]{img/Resultados/warsaw/warsaw_seg_boa.jpg}
  \caption{Segmentação boa do banco de dados \textit{\acrshort{Warsaw}}.}
\end{subfigure}\hfil % <-- added
\begin{subfigure}{0.25\textwidth}
  \includegraphics[width=\linewidth]{img/Resultados/warsaw/warsaw_seg_ruim.jpg}
  \caption{Segmentação ruim do banco de dados \textit{\acrshort{Warsaw}}.}
\end{subfigure}
\caption{Resultados bons e ruins das segmentações de íris dos bancos de dados \textit{MICHE, UBIRISv1} e \textit{\acrshort{Warsaw}}.}
\label{fig:experimentos:segmentacoes:todos}
\end{figure}

\begin{figure}[h!]
\begin{subfigure}{.25\textwidth}
\centering
\includegraphics[width=\linewidth]{img/Resultados/ubirisv2/ubirisv2_seg_boa.jpg}
\caption{Íris corretamente segmentada.}
\end{subfigure}\hfill
\begin{subfigure}{.25\textwidth}
\centering
\includegraphics[width=\linewidth]{img/Resultados/ubirisv2/ubirisv2_seg_ruim1.jpg}
\caption{Imagem onde a segmentação da pupila apresentou erros.}
\end{subfigure}\hfill
\begin{subfigure}{.25\textwidth}
\centering
\includegraphics[width=\linewidth]{img/Resultados/ubirisv2/ubirisv2_seg_ruim2.jpg}
\caption{Imagem em que tanto a pupila quanto a íris foram incorretamente segmentadas.}
\end{subfigure}
\caption{Resultados da segmentação do banco de dados \textit{UBIRISv2}.}
\label{fig:experimentos:miche_boas}
\end{figure}

\FloatBarrier

\section{Experimentos de Reconhecimento de Íris}\label{sec:experimentos:experimentos}

\par Dez etapas foram seguidas para a realização dos experimentos em cada um dos bancos de dados, e consistiram em:

\begin{enumerate}
    \item Gerar um arquivo texto com todas imagens (\textit{Todas\_Imagens});
    \item Executar a métrica \textit{\acrshort{DSMI}} em todas as imagens no arquivo texto;
    \item Aplicar o limiar $T_{DSMI}$ nas métricas \textit{\acrshort{DSMI}} calculadas e gerar um arquivo texto somente com as imagens com qualidade acima do limiar (\textit{Imagens\_DSMI});
    \item Segmentar, normalizar e codificar todas as íris das imagens;
    \item Executar a métrica \textit{\acrshort{FCE}} nos resultados da segmentação de todas as imagens;
    \item Aplicar o limiar $T_{FCE}$ nas métricas \textit{\acrshort{FCE}} calculadas e gerar um arquivo texto somente com as imagens com qualidade acima do limiar (\textit{Imagens\_FCE});
    \item Verificar as imagens que passaram nas duas métricas e gerar um arquivo texto com essas imagens (\textit{Imagens\_DSMI\_FCE});
    \item Gerar os arquivos texto que comparam uma imagem com todas as outras a partir dos arquivos \textit{Todas\_Imagens}, \textit{Imagens\_DSMI}, \textit{Imagens\_FCE} e \textit{Imagens\_DSMI\_FCE};
    \item Executar o algoritmo de correspondência do \textit{OSIRISv4.1} nos modelos gerados para as imagens dos arquivos texto da etapa 8;
    \item Aplicar as métricas de rendimento de sistemas de reconhecimento de íris nos resultados do algoritmo de correspondência do \textit{OSIRISv4.1}.
\end{enumerate}

\par Para analisar o rendimento da arquitetura de sistema de reconhecimento de íris proposta, os bancos de dados foram dividos em duas categorias:

\begin{itemize}
    \item \acrfull{SR}: somente com as três imagens de teste de cada indivíduo dos bancos de dados;
    \item \acrfull{CR}: com as três imagens de teste e as geradas com ruídos (Seção \ref{sec:experimentos:ruidos}).
\end{itemize}

\par Para as duas categorias listadas acima, foram realizados quatro experimentos para avaliar o rendimento de sistemas de reconhecimento de íris de forma a comparar com o sistema proposto:

\begin{itemize}
    \item \acrfull{SM}: sistema padrão, sem usar métricas de qualidade;
    \item Usando somente a métrica de qualidade DSMI;
    \item Usando somente a métrica de qualidade FCE;
    \item \acrfull{DM}: Arquitetura proposta no projeto (\refFig{fig:metodologia:arquitetura}).
\end{itemize}

\par Para avaliar o desempenho dos quatro experimentos, as seguintes métricas foram utilizadas:

\begin{itemize}
    \item \textit{\acrfull{AUC}} e \textit{\acrfull{ROC}} \cite{d33BEAT, aucROC, daugman2000};
    \item \textit{\acrfull{EER}} \cite{eer,d33BEAT};
    \item \textit{\acrfull{d'}} \cite{daugman2000}.
\end{itemize}

\subsection{\textit{\acrfull{AUC}} e \textit{\acrfull{ROC}}}\label{sec:experimentos:auc}

\par Sistemas biométricos devem aceitar ou rejeitar um indivíduo a partir de comparações de seus modelos com modelos armazenados \cite{wayman2005biometric}. A decisão de aceitar ou rejeitar um indivíduo pode resultar em quatro resultados: o indivíduo é corretamente aceito ou rejeitado ou é incorretamente aceito ou rejeitado. O limiar de decisão do sistema biométrico influencia diretamente nesses resultados. De forma a medir a acurácia de sistemas biométricos, quatro taxas são usadas \cite{daugman2000}:

\begin{enumerate}
    \item \textit{\acrfull{TPR}}: Taxa de indivíduos que são corretamente aceitos pelo sistema;
    \item \textit{\acrfull{TNR}}: Taxa de indivíduos que são corretamente rejeitados pelo sistema;
    \item \textit{\acrfull{FPR}}: Taxa de indivíduos que são incorretamente aceitos pelo sistema;
    \item \textit{\acrfull{FNR}}: Taxa de modeindivíduoslos que são incorretamente rejeitados pelo sistema.
\end{enumerate}

\par \textit{\acrshort{ROC}} é um gráfico comumente usado para avaliar o desempenho de sistemas de reconhecimento de íris \cite{aucROC, daugman2000}. O gráfico consiste na plotagem da \textit{\acrshort{TPR}} no eixo das ordenadas pela \textit{\acrshort{FPR}} no eixo das abcissas, que são calculadas para vários limiares de decisão. \textit{\acrshort{AUC}} é a métrica usada para avaliar o desempenho do sistema biométrico a partir do gráfico \textit{\acrshort{ROC}} e consiste em calcular a área sobre a curva do \textit{\acrshort{ROC}} \cite{aucROC}. A \textit{\acrshort{AUC}} indica quão bem o sistema biométrico sendo testado aceita e rejeita indivíduos. Pode resultar em valores entre 0.5 e 1, onde 0.5 é o pior e 1 o melhor resultado, respectivamente.

\par A \refFig{fig:experimentos:roc_semruidos} ilustra os gráficos \textit{\acrshort{ROC}} e suas \textit{\acrshort{AUC}} calculadas para os experimentos com imagens sem ruídos e a \refFig{fig:experimentos:roc_comruidos} com imagens ruidosas. Nos gráficos, \textit{Sem} são as \textit{\acrshort{AUC}} que foram calculadas sem nenhuma métrica de qualidade e \textit{Ambas} significa \textit{\acrshort{AUC}} calculadas com as duas métricas de qualidade.

\begin{figure}[h!]
    \centering % <-- added
\begin{subfigure}{0.35\textwidth}
  \includegraphics[width=\linewidth]{img/Resultados/miche_inter_nodistortion_auc.png}
  \caption{\textit{MICHE}.}
\end{subfigure}\hfil % <-- added
\begin{subfigure}{0.35\textwidth}
  \includegraphics[width=\linewidth]{img/Resultados/ubirisv1_inter_nodistortion_auc.png}
  \caption{\textit{UBIRISv1}.}
\end{subfigure}

\medskip
\begin{subfigure}{0.35\textwidth}
  \includegraphics[width=\linewidth]{img/Resultados/ubirisv2_inter_nodistortion_auc.png}
  \caption{\textit{UBIRISv2}.}
\end{subfigure}\hfil % <-- added
\begin{subfigure}{0.35\textwidth}
  \includegraphics[width=\linewidth]{img/Resultados/warsaw_inter_nodistortion_auc.png}
  \caption{{\acrshort{Warsaw}}.}
\end{subfigure}\hfil % <-- added
\caption{Curvas \textit{\acrshort{ROC}} e \textit{\acrshort{AUC}} calculados para imagens sem ruídos.}
\label{fig:experimentos:roc_semruidos}
\end{figure}

\begin{figure}[h!]
    \centering % <-- added
\begin{subfigure}{0.35\textwidth}
  \includegraphics[width=\linewidth]{img/Resultados/miche_inter_distortion_auc.png}
  \caption{\textit{MICHE}.}
\end{subfigure}\hfil % <-- added
\begin{subfigure}{0.35\textwidth}
  \includegraphics[width=\linewidth]{img/Resultados/ubirisv1_inter_distortion_auc.png}
  \caption{\textit{UBIRISv1}.}
\end{subfigure}

\medskip
\begin{subfigure}{0.35\textwidth}
  \includegraphics[width=\linewidth]{img/Resultados/ubirisv2_inter_distortion_auc.png}
  \caption{\textit{UBIRISv2}.}
\end{subfigure}\hfil % <-- added
\begin{subfigure}{0.35\textwidth}
  \includegraphics[width=\linewidth]{img/Resultados/warsaw_inter_distortion_auc.png}
  \caption{{\acrshort{Warsaw}}.}
\end{subfigure}\hfil % <-- added
\caption{Curvas \textit{\acrshort{ROC}} e \textit{\acrshort{AUC}} calculados para imagens com ruídos.}
\label{fig:experimentos:roc_comruidos}
\end{figure}

\par No banco de dados \textit{MICHE}, a arquitetura proposta com as duas métricas de qualidade não apresentou os melhores resultados tanto nos experimentos sem ruído e com ruído. Sem ruídos, usando somente a métrica \textit{\acrshort{DSMI}} apresentou a melhor \textit{\acrshort{AUC}}, enquanto com ruídos, usando somente a métrica \textit{\acrshort{FCE}} obteve a melhor \textit{\acrshort{AUC}}. Mas os dois resultados podem somente ser considerados justos \cite{aucROC}, o que confirma o desafio do banco de dados \textit{MICHE} em sistemas de reconhecimento de íris \cite{marsico2017-MICHE-1}.

\par No banco de dados \textit{\acrshort{Warsaw}}, nos testes com ruídos, a arquitetura proposta com as duas métricas obteve a melhor \textit{\acrshort{AUC}}. Já nos experimentos sem ruídos, as \textit{\acrshort{AUC}} calculadas usando somente a métrica \textit{\acrshort{FCE}} e usando as duas métricas de qualidade obtiveram um resultado praticamente perfeito, sendo 0.999 e 0.998 respectivamente. Apesar da diferença mínima, a arquitetura proposta não obteve o melhor resultado nos testes sem ruídos. Os resultados encontrados com a arquitetura proposta nos dois experimentos são excelentes \cite{aucROC}, porque apresentam \textit{\acrshort{AUC}} nos intervalos 0.9 e 1.

\par Nos bancos de dados \textit{UBIRISv1} e \textit{UBIRISv2}, a arquitetura proposta obteve os melhores resultados nos dois experimentos, com e sem ruídos. As \textit{\acrshort{AUC}} calculadas no \textit{UBIRISv1} foram excelentes assim como no \textit{\acrshort{Warsaw}} e apresentaram valores entre 0.9 e 1\cite{aucROC}. E apesar da arquitetura proposta ter apresentado o melhor resultado nos dois experimentos no \textit{UBIRISv2}, as \textit{\acrshort{AUC}} calculadas foram pobres \cite{aucROC}, apresentando valores entre 0.6 e 0.7. 

\FloatBarrier

\subsection{\textit{\acrfull{EER}}} \label{sec:experimentos:eer}

\par \textit{\acrshort{EER}} é uma métrica utilizada para calcular o desempenho de sistemas biométricos \cite{eer}. É calculada, assim como \textit{\acrshort{AUC}}, com base nas taxas \textit{\acrshort{TPR}}, \textit{\acrshort{TNR}}, \textit{\acrshort{FPR}} e \textit{\acrshort{FNR}} resultantes do processo de decisão do sistema. A taxa \textit{\acrshort{EER}} consiste no ponto em que as taxas de erro \textit{\acrshort{FPR}} e \textit{\acrshort{FNR}} são iguais, conforme as equações \cite{d33BEAT}:

\equacao{eq:experimentos:eer}{
    EER = FPR(T_{*}) = FNR(T_{*}),
}

\noindent onde 

\equacao{eq:experimentos:eer_T}{
    T_{*} = arg \underset{T}{min}(|FPR(T) - FNR(T)|)
}

\noindent é o limiar de decisão que melhor aproxima a igualdade da \refEq{eq:experimentos:eer}. Quanto mais próximo de 0 for a \textit{\acrshort{EER}} calculada, melhor o desempenho do sistema. Na \refTab{tab:experimentos:eer}, os \textit{\acrshort{EER}} calculados para todos os experimentos são apresentados e na \refTab{tab:experimentos:limiar_eer} os limiares $T_{FCE}$ da \refEq{eq:experimentos:eer_T}.

\begin{table}[h!]
\centering
\caption{Valores calculados para \textit{\acrshort{EER}} dos bancos de dados.}
\label{tab:experimentos:eer}
\tabcolsep=0.09cm
\begin{tabular}{l|l|l|l|l|l|l|l|l|}
\cline{2-9}
 & \multicolumn{2}{c|}{\textit{\textbf{MICHE}}} & \multicolumn{2}{c|}{\textit{\textbf{UBIRISv1}}} & \multicolumn{2}{c|}{\textit{\textbf{UBIRISv2}}} & \multicolumn{2}{c|}{\textit{\textbf{Warsaw}}} \\ \cline{2-9} 
 & \multicolumn{1}{c|}{\textbf{SR}} & \multicolumn{1}{c|}{\textbf{CR}} & \multicolumn{1}{c|}{\textbf{SR}} & \multicolumn{1}{c|}{\textbf{CR}} & \multicolumn{1}{c|}{\textbf{SR}} & \multicolumn{1}{c|}{\textbf{CR}} & \multicolumn{1}{c|}{\textbf{SR}} & \multicolumn{1}{c|}{\textbf{CR}} \\ \hline
\multicolumn{1}{|l|}{\textbf{SM}} & 0.367 & 0.379 & 0.357 & 0.388 & 0.428 & 0.416 & 0.058 & 0.180 \\ \hline
\multicolumn{1}{|l|}{\textbf{DSMI}} & 0.388 & \textbf{0.370} & 0.327 & 0.327 & 0.435 & 0.408 & 0.063 & 0.095 \\ \hline
\multicolumn{1}{|l|}{\textbf{FCE}} & \textbf{0.333} & 0.394 & 0.247 & 0.355 & 0.386 & 0.407 & \textbf{0.018} & 0.105 \\ \hline
\multicolumn{1}{|l|}{\textbf{DM}} & 0.341 & 0.392 & \textbf{0.097} & \textbf{0.097} & \textbf{0.381} & \textbf{0.394} & 0.028 & \textbf{0.023} \\ \hline
\end{tabular}
\end{table}

\par As melhores \textit{\acrshort{EER}} calculadas para o banco de dados \textit{MICHE} foram usando somente as métrica \textit{\acrshort{FCE}} e \textit{\acrshort{DSMI}} nos experimentos sem ruído e com ruído, respectivamente. Esses valores, assim como os encontrados para \textit{\acrshort{AUC}}, são valores ruins, porque quanto maior a \textit{\acrshort{EER}}, pior o desempenho do sistema \cite{eer, d33BEAT} e reafirmam a dificuldade do  \textit{MICHE}.

\par Assim como nos testes da \textit{\acrshort{AUC}}, as melhores \textit{\acrshort{EER}} do banco de dados \textit{\acrshort{Warsaw}} sem e com ruídos foram calculadas usando somente a métrica \textit{\acrshort{FCE}} e na arquitetura proposta. Os avalores também são excelentes, porque são próximos de 0 \cite{eer, d33BEAT}.

\par A arquitetura proposta, assim como na métrica \textit{\acrshort{AUC}}, apresentou os melhores resultados nos experimentos sem e com ruídos nos bancos de dados \textit{UBIRISv1} e \textit{UBIRISv2}. E assim como os resultados de \textit{\acrshort{AUC}}, os valores apresentados no \textit{UBIRISv2} são considerados ruins, porque são \textit{\acrshort{EER}} elevados e no \textit{UBIRISv1} foram excelentes, com valores iguais e perto de 0 \cite{eer, d33BEAT}.

\begin{table}[h!]
\centering
\caption{Limiares de decisão encontrados a partir do \textit{\acrshort{EER}} ($T_{FCE}$).}
\label{tab:experimentos:limiar_eer}
\tabcolsep=0.09cm
\begin{tabular}{l|l|l|l|l|l|l|l|l|}
\cline{2-9}
 & \multicolumn{2}{c|}{\textit{\textbf{MICHE}}} & \multicolumn{2}{c|}{\textit{\textbf{UBIRISv1}}} & \multicolumn{2}{c|}{\textit{\textbf{UBIRISv2}}} & \multicolumn{2}{c|}{\textit{\textbf{Warsaw}}} \\ \cline{2-9} 
 & \multicolumn{1}{c|}{\textbf{SR}} & \multicolumn{1}{c|}{\textbf{CR}} & \multicolumn{1}{c|}{\textbf{SR}} & \multicolumn{1}{c|}{\textbf{CR}} & \multicolumn{1}{c|}{\textbf{SR}} & \multicolumn{1}{c|}{\textbf{CR}} & \multicolumn{1}{c|}{\textbf{SR}} & \multicolumn{1}{c|}{\textbf{CR}} \\ \hline
\multicolumn{1}{|l|}{\textbf{SM}} & 0.424020 & 0.431624 & 0.431900 & 0.433628 & 0.426407 & 0.429078 & 0.408257 & 0.428270 \\ \hline
\multicolumn{1}{|l|}{\textbf{DSMI}} & 0.425373 & 0.431090 & 0.428070 & 0.428070 & 0.429293 & 0.430233 & 0.410030 & 0.419355 \\ \hline
\multicolumn{1}{|l|}{\textbf{FCE}} & 0.426117 & 0.441176 & 0.425926 & 0.437984 & 0.423810 & 0.441667 & 0.404762 & 0.427305 \\ \hline
\multicolumn{1}{|l|}{\textbf{DM}} & 0.430556 & 0.439655 & 0.406250 & 0.406250 & 0.427083 & 0.438172 & 0.407986 & 0.40833 \\ \hline
\end{tabular}
\end{table}

\FloatBarrier

\subsection{\textit{\acrfull{d'}}}\label{sec:experimentos:daugman}

\par O limiar de decisão do sistema biométrico que influencia nas taxa de acerto e erro nas correspondências. Quanto mais conservador, maior a \textit{\acrshort{FNR}} e quanto menos criterioso, maior a \textit{\acrshort{FPR}} \cite{daugman2000}. A quantidade de sobreposição das distribuições de aceitos e rejeitados que indica o quão boa a taxa de decisão do sistema é, porque indicam as taxas \textit{\acrshort{FPR}} e \textit{\acrshort{FNR}}. A métrica \textit{\acrshort{d'}} foi proposta de forma a calcular a separação da distribuição das íris aceitas e rejeitadas, o grau de detrimento entre as duas taxas de erro e quantificar em um valor a capacidade de decisão de um sistema de reconhecimento de íris, e é descrita pela equação \cite{daugman2000}:

\equacao{eq:experimentos:daugman}{
    d' = \frac{|\mu_{1} - \mu_{2}|}{\sqrt{\frac{1}{2}(\sigma_{1}^2 + \sigma_{2}^2)}},
}

\noindent onde $\mu_{1}$ e $\mu_{2}$ são as médias das \textit{\acrshort{HD}} das íris que foram aceitas e rejeitadas, respectivamente, enquanto $\sigma_{1}^2$ e $\sigma_{2}^2$ são seus desvios padrões. Quanto maior o valor de \textit{\acrshort{d'}}, melhor o poder de decisão do sistema de reconhecimento de íris.

\par Nos experimentos realizados, \textit{\acrshort{d'}} foi calculado utilizando quatro limiares de decisão: $T_{1} = 0.2$, $T_{2} = 0.3$, $T_{3} = 0.4$ e $T_{EER}$. $T_{EER}$ consiste nos limiares encontrados na \textit{\acrshort{EER}}, e são enumerados na \refTab{tab:experimentos:limiar_eer}.

% \begin{table}[h!]
% \centering
% \caption{Limiares utilizados para o cálculo do \textit{\acrshort{d'}}}
% \label{tab:experimentos:limiares}
% \begin{tabular}{l|l|l|l|}
% \cline{2-4}
%  & $T_{1}$ & $T_{2}$ & $T_{3}$ \\ \hline
% \multicolumn{1}{|l|}{\textbf{Limiar}} & 0.2 & 0.3 & 0.4 \\ \hline
% \end{tabular}
% \end{table}


\begin{table}[h!]
\centering
\caption{Valores calculados para \textit{\acrshort{d'}} nas imagens sem ruídos.}
\label{tab:experimentos:d:sem_ruido}
\tabcolsep=0.02cm
\begin{tabular}{l|l|l|l|l|l|l|l|l|l|l|l|l|l|l|l|l|}
\cline{2-17}
 & \multicolumn{4}{c|}{\textit{\textbf{MICHE}}} & \multicolumn{4}{c|}{\textit{\textbf{UBIRISv1}}} & \multicolumn{4}{c|}{\textit{\textbf{UBIRISv2}}} & \multicolumn{4}{c|}{\textit{\textbf{Warsaw}}} \\
 \cline{2-17} &
\multicolumn{1}{c|} {\textbf{$T_{1}$}} & \textbf{$T_{2}$} & \textbf{$T_{3}$} & \textbf{$T_{EER}$} & \textbf{$T_{1}$} & \textbf{$T_{2}$} & \textbf{$T_{3}$} & \textbf{$T_{EER}$} & \textbf{$T_{1}$} & \textbf{$T_{2}$} & \textbf{$T_{3}$} & \textbf{$T_{EER}$} & \textbf{$T_{1}$} & \textbf{$T_{2}$} & \textbf{$T_{3}$} & \textbf{$T_{EER}$} \\ \hline
\multicolumn{1}{|l|}{\textbf{SM}} & 5.78 & 3.40 & \textbf{2.35} & \textbf{2.23} & 7.19 & 3.26 & \textbf{2.26} & 1.85 & 4.43 & 2.73 & 1.81 & 1.67 & 10.49 & 7.04 & 1.84 & 1.66 \\ \hline
\multicolumn{1}{|l|}{\textbf{DSMI}} & 5.75 & 3.56 & 2.19 & 2.10 & 6.74 & 4.16 & 2.21 & \textbf{1.86} & 4.37 & 2.65 & 1.71 & 1.56 & 10.46 & 6.66 & 1.89 & 1.62 \\ \hline
\multicolumn{1}{|l|}{\textbf{FCE}} & 6.87 & 3.97 & 1.99 & 2.01 & \textbf{9.96} & 6.06 & 1.66 & 1.29 & \textbf{16.48} & 5.65 & 2.73 & \textbf{2.65} & \textbf{11.44} & \textbf{8.83} & \textbf{2.44} & \textbf{2.03} \\ \hline
\multicolumn{1}{|l|}{\textbf{DM}} & \textbf{7.68} & \textbf{4.22} & 1.70 & 1.68 & 9.42 & \textbf{6.49} & 1.74 & 1.55 & 16.26 & \textbf{6.48} & \textbf{2.84} & 2.63 & 11.04 & 8.09 & 2.29 & 1.79 \\ \hline
\end{tabular}
\end{table}

%Isso pode ser explicado porque quanto menor os limiares, mais criterioso é o sistema, e menos correspondência são aceitas, de forma que a diferença absoluta de suas médias é maior

\begin{table}[h!]
\centering
\caption{Valores calculados para \textit{\acrshort{d'}} nas imagens com ruídos.}
\label{tab:experimentos:d:com_ruido}
\tabcolsep=0.04cm
\begin{tabular}{l|l|l|l|l|l|l|l|l|l|l|l|l|l|l|l|l|}
\cline{2-17}
 & \multicolumn{4}{c|}{\textit{\textbf{MICHE}}} & \multicolumn{4}{c|}{\textit{\textbf{UBIRISv1}}} & \multicolumn{4}{c|}{\textit{\textbf{UBIRISv2}}} & \multicolumn{4}{c|}{\textit{\textbf{Warsaw}}} \\
 \cline{2-17} &
\multicolumn{1}{c|} {\textbf{$T_{1}$}} & \textbf{$T_{2}$} & \textbf{$T_{3}$} & \textbf{$T_{EER}$} & \textbf{$T_{1}$} & \textbf{$T_{2}$} & \textbf{$T_{3}$} & \textbf{$T_{EER}$} & \textbf{$T_{1}$} & \textbf{$T_{2}$} & \textbf{$T_{3}$} & \textbf{$T_{EER}$} & \textbf{$T_{1}$} & \textbf{$T_{2}$} & \textbf{$T_{3}$} & \textbf{$T_{EER}$} \\ \hline
\multicolumn{1}{|l|}{\textbf{SM}} & 5.70 & 3.48 & 2.13 & 1.94 & 6.57 & 3.86 & \textbf{2.29} & \textbf{1.92} & 3.99 & 2.67 & 1.68 & 1.48 & 7.38 & 4.70 & 1.88 & 1.42 \\ \hline
\multicolumn{1}{|l|}{\textbf{DSMI}} & 6.10 & 3.86 & 2.34 & 2.17 & 6.74 & 4.16 & 2.21 & 1.86 & 4.09 & 2.67 & 1.62 & 1.41 & 7.62 & 4.82 & 2.06 & 1.56 \\ \hline
\multicolumn{1}{|l|}{\textbf{FCE}} & \textbf{7.24} & \textbf{4.13} & 1.91 & 1.82 & 8.33 & 5.06 & 1.72 & 1.45 & \textbf{8.10} & \textbf{3.96} & 2.39 & 2.26 & \textbf{8.43} & \textbf{5.16} & 2.26 & 1.31 \\ \hline
\multicolumn{1}{|l|}{\textbf{DM}} & 6.51 & 4.10 & \textbf{2.37} & \textbf{2.18} & \textbf{9.42} & \textbf{6.49} & 1.74 & 1.55 & 7.29 & 3.86 & \textbf{2.61} & \textbf{2.42} & 7.82 & 5.04 & \textbf{2.44} & \textbf{1.95} \\ \hline
\end{tabular}
\end{table}

\par Tanto nos experimentos sem ruído quanto com ruído, os valores do \textit{\acrshort{d'}} calculados apresentaram os melhores resultados com os menores limiares. Isso pode ser explicado porque quanto menor o limiar, menor a média dos valores da \textit{\acrshort{HD}} das comparações aceitas, e como a média das \textit{\acrshort{HD}} das comparações rejeitadas mantém-se praticamente constante, o numerador da \refEq{eq:experimentos:daugman} aumenta e os seus resultados consequentemente. As Tabelas \ref{tab:experimentos:media_sem_ruidos_daugman} e \ref{tab:experimentos:media_com_ruidos_daugman} no Apêndice \ref{Apendice_B} enumeram as médias, variâncias e quantidade de comparações aceitas e rejeitadas calculadas para os experimentos sem ruído.


\par Nos experimentos sem ruído, a arquitetura proposta com as duas métricas de qualidade apresentaram os melhores resultados do índice \textit{\acrshort{d'}} com os limiares $T_{1}$ e $T_{2}$ no banco de dados \textit{MICHE}. O banco de dados \textit{UBIRISv1} apresentou o melhor valor de \textit{\acrshort{d'}} com a arquitetura proposta somente com o limiar $T_{2}$. No banco de dados \textit{UBIRISv2}, a arquitetura proposta obteve os maiores valores no experimento com os limiares $T_{2}$ e $T_{3}$. Por fim, no banco de dados \textit{\acrshort{Warsaw}}, a arquitetura proposta, contrariando os bons resultados obtidos com as métricas \textit{\acrshort{AUC}} e \textit{\acrshort{EER}}, não obteve o melhor resultado com nenhum dos limiares.

\par Já nos experimentos com ruídos, o banco de dados \textit{MICHE} apresentou os maiores índices com a arquitetura proposta com os limiares $T_{3}$ e $T_{EER}$. No \textit{UBIRISv1}, os dois maiores valores com a arquitetura proposta foram calculados com dois limiares, ao invés de somente um como nos experimenos sem ruído: $T_{1}$ e $T_{2}$. O banco de dados \textit{UBIRISv2} também apresentou dois maiores valores com a arquitetura proposta como no experimento sem ruído, mas com os limiares $T_{2}$ e $T_{3}$. Os resultados encontrados com a arquitetura proposta no \textit{\acrshort{Warsaw}} foram melhores no experimento com ruídos, onde em dois limiares foi encontrado os maiores valores no índice, $T_{3}$ e $T_{EER}$, ao invés de nenhum.

\par Os resultados encontrados na arquitetura proposta usando a métrica \textit{\acrshort{d'}} nos experimentos sem e com ruídos, apesar de não terem sido os melhores com todos os limiares nos bancos de dados, apresentaram bons valores. Os valores menores podem ser explicados pela variância da \textit{\acrshort{HD}} calculada com a arquitetura proposta. Como a quantidade de imagens filtradas pelas duas métricas é menor, a sua variância aumenta e quanto maior a variância, com as médias mantendo valores parecidos, maior o denominador da \refEq{eq:experimentos:daugman} e menor o valor do índice calculado.

\par No próximo capítulo, as conclusões quanto à arquitetura proposta com as duas métricas de qualidade  são apresentadas.
