
\par Este capítulo descreve os métodos para a implementação do projeto proposto. São mostrados o diagrama da arquitetura do sistema de reconhecimento de íris com as métricas \textit{\acrshort{DSMI}} e \textit{\acrshort{FCE}}, e cada um dos módulos ou blocos desse diagrama.

% Ruídos gerados
% Ver slides das aulas 7,8 e 9 de IPI, e ver https://en.wikipedia.org/wiki/Image_noise

\section{Arquitetura proposta} \label{sec:metodologia:arquitetura}

\figura[h!]{img/Metodologia/arquitetura_proposta}{Diagrama da arquitetura de sistema de reconhecimento de íris com as métricas de qualidade\textit{\acrshort{DSMI}} e \textit{\acrshort{FCE}} proposta. $T_{DSMI}$ é o limiar definido para a métrica \textit{DSMI}, $T_{FCE}$ o limiar da métrica \textit{FCE} e \textit{T} o limiar para aceitar ou rejeitar um indivíduo que está sendo autenticado a partir da \textit{\acrfull{HD}} calculada entre seus modelos}{fig:metodologia:arquitetura}{width=0.8\textwidth}

\par A \refFig{fig:metodologia:arquitetura} ilustra o diagrama da arquitetura proposta. Conforme explicitado na \refFig{fig:metodologia:arquitetura}, o funcionamento da arquitetura proposta consiste em: proporcionar uma imagem de íris \textit{\acrfull{LV}} de entrada; calcular a sua qualidade $Q_{DSMI}$ por meio da métrica \textit{\acrshort{DSMI}} e verificar por meio do limiar $T_{DSMI}$ se a qualidade é boa o bastante, se sim, permitir que a imagem para o módulo de segmentação, se não solicitar por outra imagem de íris; segmentar e normalizar a imagem da íris por algum algoritmo de segmentação e normalização; calcular a qualidade da segmentação $Q_{FCE}$ e verificar se é ou não boa o bastante para a codificação a partir do limiar $T_{FCE}$, se sim, permitir que a íris seja codificada e, se não, solicitar por outra imagem de íris e repetir as etapas; extrair os atributos da íris segmentada e gerar seu código ou modelo (\textit{IrisCode}) por algum algoritmo de codificação; comparar o modelo calculado com modelos armazenados no módulo de correspondência, que podem ou não ser do mesmo indivíduo, dependendo do tipo de sistema de reconhecimento de íris, e calcular a \textit{\acrfull{HD}} dos modelos; e por fim, comparar a \textit{\acrshort{HD}} calculada com o limiar \textit{T} definido para o sistema e verificar se houve correspondência e aceitar ou rejeitar a autenticação.

\par A arquitetura proposta foi projetada para ser flexível, de forma que qualquer sistema de reconhecimento de íris pudesse ser usado. Essa flexibilidade possibilita o uso de diversos algoritmos de segmentação, codificação e correspondência de íris, contanto que suas saídas sejam compatíveis com as métricas \textit{\acrshort{DSMI}} e \textit{\acrshort{FCE}}, e que o resultado da correspondência seja uma distância.

\par Como o objetivo do projeto é a análise das métricas na taxa de desempenho de sistemas de reconhecimento de íris, o sistema de reconhecimento de íris \textit{OSIRISv4.1} \cite{othman2015, osirisv41, osirisv41_doc} foi utilizado na arquitetura proposta, e a etapa de registro foi desconsiderada no diagrama da \refFig{fig:metodologia:arquitetura}, já que somente a análise da etapa de autenticação é necessária porque foram utilizados bancos de dados de imagens de íris. Para utilizar o \textit{OSIRISv4.1}, devem ser criados quatro arquivos de configurações \cite{osirisv41_doc}:

\begin{enumerate}
    \item Segmentação;
    \item Correspondência;
    \item Pontos de aplicação;
    \item Filtros de \textit{Gabor}.
\end{enumerate}

%, apresentar \textit{\acrfull{EER}} de 0.001 e 0.035 \cite{othman2015} nas bases de dados \textit{ICE 2005} \cite{ice2005} e \textit{CASIA-IrisV4} \cite{casiav4}, respectivamente,

% \par Por ser um sistema de código aberto (\textit{open source}) e ser o sistema usado pelos autores da métrica \textit{\acrshort{DSMI}} \cite{Jenadeleh_2018_CVPR_Workshops} para seus experimentos, o sistema de reconhecimento de íris \textit{OSIRISv4.1} foi escolhido para o sistema da arquitetura proposta. 


\par No arquivo de segmentação devem ser passados parâmetros para o funcionamento dos algoritmos, um arquivo texto do computador que contém uma lista de imagens de íris para ser processadas, o diretório que contém as imagens e o diretório em que as saídas do \textit{OSIRISv4.1} vão ser armazenadas. O arquivo de correspondência precisa de uma lista com pares de imagens de íris que já foram processadas pelo sistema e o arquivo de texto que o resultado das comparações será salvo, que consiste na \textit{\acrshort{HD}} calculada para os pares de imagens. Os arquivos de pontos de aplicação e filtros de \textit{Gabor} são disponibilizados em \cite{osirisv41} e os autores recomendam que sejam os usados.

\par Os limiares $T_{DSMI}$, $T_{FCE}$ e $T$ foram definidos de acordo com os critérios explicados no \refCap{5_Resultados}.

Cada um dos blocos ou módulos apresentados na \refFig{fig:metodologia:arquitetura} são detalhados nas Seções \ref{sec:metodologia:dsmi} até \ref{sec:metodologia:correspondencia}.

% \par Cada um dos blocos ou módulos apresentados na \refFig{fig:metodologia:arquitetura} são explicados como funcionam, como foram implementados e, para o caso de serem módulos de reconhecimento de íris, como funcionam no \textit{OSIRISv4.1} com mais detalhes nas Seções \ref{sec:metodologia:dsmi} até \ref{sec:metodologia:correspondencia}.

\subsection{DSMI}\label{sec:metodologia:dsmi}

\par O módulo \acrshort{DSMI} consiste no módulo responsável pelo cálculo da qualidade da imagem de íris de entrada no sistema proposto, pela métrica \textit{\acrshort{DSMI}}. O módulo deve receber como entrada uma imagem de íris \textit{\acrshort{LV}}, calcular sua qualidade e decidir se a imagem está adequada para continuar para o próximo módulo. A \refFig{fig:metodologia:dsmi} ilustra o módulo e os passos descritos acima.

\figura[h!]{img/Metodologia/dsmi}{Representação do módulo da métrica de qualidade \textit{DSMI} no diagrama da arquitetura proposta. A entrada é uma imagem de íris \textit{\acrshort{LV}} e saída é a qualidade da imagem $Q_{DSMI}$, sendo um valor entre 0 e 1. Enfim, o módulo deve decidir se a imagem é ou não boa o suficiente com base no limiar $T_{DSMI}$}{fig:metodologia:dsmi}{width=0.5\textwidth}

% \par A métrica consiste no cálculo de componentes de sinal e magnitude da imagem, $S_{1}$ e $S_{2}$, na combinação desses componentes na estatística de coincidência de padrões $S$ e no cálculo da variância das vizinhanças da imagem $\sigma^2$, para enfim, calcular a qualidade final $Q_{DSMI}$.

\par O componente de sinal $S_{1}$ consiste no cálculo dos máximos locais da imagem de entrada (\refEq{eq:dsmi:s1}) e o componente de magnitude $S_{2}$ na comparação de um limiar global \textit{T} com as máximas diferenças locais (\refEq{eq:dsmi:s2}). Os passos para calcular os máximos locais e as máximas diferenças seguem os seguintes passos:

\begin{enumerate}
    \item Obter imagem resultante, nomeada $Maximas$, do processamento da imagem de entrada com a operação morfológica de dilatação (Seção \ref{sec:morf:dil}), considerando a vizinhança vertical e horizontal (elemento estruturante da \refFig{fig:kernel_morfo});
    \item Obter imagem resultante, chamada de $Minimas$, do processamento da imagem de entrada com a operação morfológica de erosão (Seção \ref{sec:morf:ero}), considerando a vizinhança vertical e horizontal (elemento estruturante da \refFig{fig:kernel_morfo});
    \item Calcular as absolutas diferenças entre as imagens $Maximas$ e a imagem de entrada, conforme: $X = |I_{entrada} - Maximas|$;
    \item Calcular as absolutas diferenças entre a imagem de entrada e a imagem $Minimas$, conforme: $Y = |Minimas - I_{entrada}|$;
    \item Comparar os \textit{pixels} da imagem de entrada com o equivalente da imagem $Maximas$, e verificar se é maior, se for, é um máximo local e 1 é atribuído à posição em $S_{1}$ (\refEq{eq:dsmi:s1});
    \item Calcular $max(X, Y)$ de forma a obter as máximas diferenças em uma matriz, chamada de $MaxDiffs$, da imagem de entrada.
\end{enumerate}

\par Com os máximos locais encontrados, o limiar $T$ usado para calcular os componentes de magnitude $S_{2}$ é calculado conforme a \refEq{eq:dsmi:T}, para então calcular $S_{2}$. 

\par A estatística da coincidência dos padrões $S$ é calculada de acordo com a \refEq{eq:dsmi:s}. A variância das vizinhanças dos \textit{pixels} da imagem de entrada $\sigma^2$ (\refEq{eq:dsmi:var}) é calculada por meio do filtro de \textit{Variância Local} (Seção \ref{sec:dom_esp:filtro_std}) com a máscara (\textit{kernel}) representada pela \refFig{fig:kernel_morfo}.

\par Com $S$ e $\sigma^2$, a qualidade final $Q_{DSMI}$ é, por fim, calculada conforme a \refEq{eq:dsmi:Q} e normalizada pela \refEq{eq:dsmi:r}. O módulo decide então se a imagem é ou não boa o suficiente para ser segmentada comparando o limiar $T_{DSMI}$ com a qualidade $Q_{DSMI}$.

% \begin{algorithm} % enter the algorithm environment
% \caption{Calculate $y = x^n$} % give the algorithm a caption
% \label{alg1} % and a label for \ref{} commands later in the document
% \begin{algorithmic} % enter the algorithmic environment
%     \REQUIRE $n \geq 0 \vee x \neq 0$
%     \ENSURE $y = x^n$
%     \STATE $y \Leftarrow 1$
%     \IF{$n < 0$}
%         \STATE $X \Leftarrow 1 / x$
%         \STATE $N \Leftarrow -n$
%     \ELSE
%         \STATE $X \Leftarrow x$
%         \STATE $N \Leftarrow n$
%     \ENDIF
%     \WHILE{$N \neq 0$}
%         \IF{$N$ is even}
%             \STATE $X \Leftarrow X \times X$
%             \STATE $N \Leftarrow N / 2$
%         \ELSE[$N$ is odd]
%             \STATE $y \Leftarrow y \times X$
%             \STATE $N \Leftarrow N - 1$
%         \ENDIF
%     \ENDWHILE
% \end{algorithmic}
% \end{algorithm}

\subsection{Segmentação}\label{sec:metodologia:segmentacao}

\par O módulo de segmentação é o módulo responsável por segmentar e normalizar a imagem de íris de entrada no sistema. O módulo deve receber a imagem de íris que passou pelo módulo \acrshort{DSMI}, segmentá-la e normalizá-la, resultando nos parâmetros: raio e posição central da íris e da pupila, onde (X, Y, R) são as coordenadas da posição e o raio, respectivamente; máscara binária da íris segmentada; íris normalizada em coordenadas polares; e a máscara binária da imagem normalizada. Nas máscaras binárias, os \textit{pixels} pretos são ruídos como pálpebras e cílios, e os \textit{pixels} brancos são íris corretamente segmentadas. A \refFig{fig:metodologia:segmentacao} ilustra o módulo e as entradas e saídas do módulo.

\figura[h!]{img/Metodologia/modulo_segmentacao}{Módulo que segmenta e normaliza a imagem de íris. As etapas de segmentação e normalização são divididas em dois blocos separados, de forma que qualquer algoritmo de segmentação ou normalização pudesse ser usado. Sua entrada consiste na imagem de íris \textit{\acrshort{LV}} e suas saídas são as posições centrais e raio da íris e pupila, máscaras binárias da íris e da íris normalizada, e a íris normalizada}{fig:metodologia:segmentacao}{width=0.8\textwidth}

\figura[h!]{img/Metodologia/Img_24_1_4}{Imagem de íris \textit{\acrshort{LV}}}{fig:metodologia:original}{width=.3\textwidth}

\par A etapa de segmentação do \textit{OSIRISv4.1} consiste em dois passos: encontrar a posição central da pupila e seu raio para então encontrar a íris \cite{osirisv41_doc}. A pupila é encontrada seguindo dois critérios: (i) calcular o valor mínimo na imagem sendo filtrada por uma máscara de raio $r$, onde o processo de filtragem é a soma da vizinhança considerada pela máscara; (ii) calcular os gradientes verticais e horizontais da imagem usando o operador \textit{Sobel} \cite{kanopoulos1988}, construir duas máscaras de raio $r$, filtrar os gradientes com as duas máscaras construídas e encontrar o valor máximo da soma do resultado das duas filtragens. Os critérios (i) e (ii)  são repetidos para vários valores de raio da pupila $r$, e o valor máximo do resultado das somas dos dois critérios representa o melhor candidato para a pupila. Os contornos da íris, então, devem ser encontrados usando o algoritmo \textit{Viterbi} \cite{sutra2012}. O algoritmo é aplicado em duas resoluções: altas e baixas. A resolução alta é usada para encontrar os contornos precisos da íris e a resolução baixa para os contornos grosseiros (\textit{coarse contours}). O centro da íris e seu raio são calculados pelos procedimentos descritos acima. O intervalo de valores dos raios que devem ser usados para procurar a íris e pupila são passados por arquivos de configuração, conforme descrito em \cite{osirisv41_doc}. Os valores utilizados neste projeto são enumerados no Capítulo 5, porque variaram para cada base de dados utilizada nos experimentos. A \refFig{fig:metodologia:segmen} ilustra os resultados da etapa de segmentação da imagem \refFig{fig:metodologia:original}.

\figurasDuplas{H}{img/Metodologia/Img_24_1_4_segm}{img/Metodologia/Img_24_1_4_mask}{Íris da \refFig{fig:metodologia:original} segmentada. Os valores encontrados para (Xp, Yp, Rp) e (Xi, Yi, Ri) foram: (419, 274, 52) e (414, 276, 183), respectivamente}{Íris e pupila segmentadas}{Máscara binária resultante da íris segmentada}{fig:metodologia:segmen}{fig:metodologia:segmen_colorida}{{fig:metodologia:segmen_binaria}{.5\textwidth}{width=.6\linewidth}}

\par A etapa de normalização do \textit{OSIRISv4.1} é feita usando os contornos grosseiros encontrados na etapa de segmentação em conjunto com o \textit{Modelo Rubber Sheet} (Seção \ref{sec:iris:rec_iris}) \cite{osirisv41_doc}. Os valores das dimensões da imagem normalizada devem ser passados por arquivos de configuração, conforme descrito em \cite{osirisv41_doc}. Os valores utilizados neste projeto são: 40 linhas e 300 colunas, ou seja, $40 \times 300$. A \refFig{fig:metodologia:normalizadas} ilustra os resultados da etapa de normalização da imagem \refFig{fig:metodologia:original}.

\figurasDuplas{H}{img/Metodologia/Img_24_1_4_imno}{img/Metodologia/Img_24_1_4_mano}{Íris da \refFig{fig:metodologia:original} normalizada}{Íris normalizada em coordenadas polares}{Máscara binária da íris normalizada}{fig:metodologia:normalizadas}{fig:metodologia:normalizada_imno}{{fig:metodologia:normalizada_mano}{.5\textwidth}{width=.8\linewidth}}

\FloatBarrier

\subsection{FCE}\label{sec:metodologia:fce}

\par O módulo \acrshort{FCE} é o módulo que é responsável por calcular a qualidade da segmentação da íris da imagem de entrada, usando a métrica de qualidade \textit{\acrshort{FCE}}. O módulo recebe os resultados do módulo de segmentação, calcula a qualidade da segmentação e decide se a segmentação foi boa, de forma a ser passada para o módulo de codificação. A \refFig{fig:metodologia:fce} ilustra as entradas e saídas do módulo \acrshort{FCE}.

\figura[H]{img/Metodologia/fce}{Módulo da métrica de qualidade \textit{\acrshort{FCE}} que calcula a qualidade da segmentação da íris. Recebe como entrada a íris e a máscara normalizadas e os raios da pupila e íris. A saída consiste na qualidade da segmentação $Q_{FCE}$, sendo um valor entre 0 e 1. O módulo deve então decidir se a segmentação da íris foi boa o suficiente para ser codificada, de acordo com o limiar $T_{FCE}$}{fig:metodologia:fce}{width=0.6\textwidth}

\par A métrica consiste no cálculo de três submétricas, conforme descrito na Seção \ref{sec:revisao:fce}:

\begin{enumerate}
    \item Métrica de Correlação dos Atributos;
    \item Métrica de oclusão;
    \item Métrica de dilatação,
\end{enumerate}

\noindent em que a métrica de correlação dos atributos ($FCM$) consiste na medição da qualidade e a distintividade dos padrões presentes na íris segmentada, métrica de oclusão ($O$) na medição da quantidade de ruídos presentes na íris segmentada e a métrica de dilatação ($D$) na medição de quanto a pupila está dilatada. As métricas são então combinadas na qualidade final da segmentação $Q_{FCE}$.

\par Para o cálculo da métrica $FCM$, é necessária a imagem da íris normalizada. Os passos descritos abaixo são feitos para cada par adjacente de linhas da imagem da íris normalizada, $\vec{l}_{i}$ e $\vec{l}_{i+1}$, conforme descrito pelas \refEqs{eq:fce:fcm1}{eq:fce:fcm4} na Seção \ref{sec:revisao:fce}:

\begin{enumerate}
    \item Filtrar as linhas $\vec{l}_{i}$ e $\vec{l}_{i+1}$ com o filtro \textit{Log-Gabor} (\refEq{eq:filtragem:loggabor});
    \item Calcular a função de massa probabilidade das linhas $\vec{l}_{i}$ e $\vec{l}_{i+1}$ filtradas (\refEqs{eq:fce:fcm1}{eq:fce:fcm2});
    \item Calcular a distância de informação $J_{i}$ com as entropias relativas e armazená-la em um vetor (\refEq{eq:fce:fcm3});
\end{enumerate}

\par Quando os passos enumerados acima forem repetidos para todas as linhas da imagem de íris normalizada, $FCM$ é calculado pela média das distâncias de informaçãp $J_{i}$ (\refEq{eq:fce:fcm5}). Os parâmetros utilizadas no filtro Log-Gabor na \refEq{eq:filtragem:loggabor} são: $\omega_{0} = 22$ e $\sigma = 0.5$. Para o cálculo da métrica final, $FCM$ é normalizado pela função $f()$ (\refEq{eq:fce:normFCM}). Como a métrica foi proposta para imagens no espectro \textit{NIR} \cite{du2010}, e o projeto proposto trabalha com imagens no espectro de luz visível, o parâmetro $\beta$ foi recalculado com imagens de treino dos bancos de imagens utilizados no projeto, de forma que experimentalmente $\beta = 0.294$ foi obtido. Já o parâmetro $\alpha$ continua sendo $\alpha = 1/\beta$.

\par A métrica de oclusão $O$ precisa da máscara binária normalizada da íris para ser calculada e o cálculo consiste em somar todos os \textit{pixels} pretos e dividir pela quantidade total de \textit{pixels} da imagem, conforme a \refEq{eq:fce:O}. A métrica $O$ deve então ser normalizada pela função $g()$ (\refEq{eq:fce:normO}).

\par A métrica de dilatação $D$ precisa dos raios da pupila e íris e o seu cálculo consiste simplesmente em aplicar a \refEq{eq:fce:D}. A função de normalização $h()$ (\refEq{eq:fce:normD}) é então aplicada.

\par Enfim, a qualidade final $Q_{FCE}$ é calculada pela \refEq{eq:fce:Q} e o módulo decide se a íris segmentada é apropriada para o módulo de codificação, comparando o limiar $T_{FCE}$ com a qualidade $Q_{FCE}$.

\subsection{Codificação}\label{sec:metodologia:codificacao}

\par O módulo de codificação é o módulo responsável por codificar os atributos extraídos da íris segmentada e gerar o código da íris (\textit{IrisCode}). O módulo deve receber a íris normalizada em coordenadas polares, resultado do módulo de segmentação, e codificá-la por meio de algum algoritmo e gerar o \textit{IrisCode} correspondente. A \refFig{fig:metodologia:codificacao} ilustra o módulo e as suas entradas e saídas.

\figura[h!]{img/Metodologia/modulo_codificacao}{Módulo que codifica a íris segmentada. As etapas de extração dos atributos e codificação são divididas em dois submódulos. Sua entrada consiste na íris normalizada em coordenadas polares, resultado do módulo de segmentação, e sua saída é o código da íris (\textit{IrisCode})}{fig:metodologia:codificacao}{width=0.8\textwidth}

\par A etapa de codificação do \textit{OSIRISv4.1} consiste na aplicação de três filtros \textit{Gabor}. Porém, como a parte real e imaginária são consideradas independentemente, acabam sendo seis processos de filtragem \cite{osirisv41_doc}. Os resultados das seis filtragens passam pelo processo das \textit{Wavelets 2D de Gabor} (Seção \ref{sec:iris:rec_iris}), de forma que 6 imagens binárias, ou 6 códigos de íris, são combinados em um único \textit{IrisCode}. A \refFig{fig:metodologia:iriscode} ilustra o \textit{IrisCode} gerado para a íris da \refFig{fig:metodologia:original}. Os filtros de \textit{Gabor} utilizados são fornecidos em \cite{osirisv41} pelos autores no formato de um arquivo de configuração.

\figura[H]{img/Metodologia/Img_24_1_4_code}{\textit{IrisCode} da \refFig{fig:metodologia:original} gerado pelo \textit{OSIRISv4.1}. Pode-se notar que são 6 códigos de íris combinados em um só, para cada um dos filtros \textit{Gabor} aplicados}{fig:metodologia:iriscode}{width=0.5\textwidth}

\subsection{Correspondência}\label{sec:metodologia:correspondencia}

\par O módulo de correspondência é o último módulo do sistema proposto e é responsável por calcular a distância entre dois códigos de íris e decidir se o sistema deve ou não aceitar a autenticação. O módulo deve receber dois códigos de íris e as máscaras binárias normalizadas correspondentes e usá-los para calcular a \textit{\acrfull{HD}} (Seção \ref{sec:iris:rec_iris}). A \refFig{fig:metodologia:correspondencia} ilustra o módulo e as suas entradas e saídas.

\figura[h!]{img/Metodologia/modulo_correspondencia}{Módulo que calcula a \textit{\acrfull{HD}} entre dois modelos e decide se correspondem ou não a mesma pessoa. O módulo recebe como entrada dois \textit{IrisCodes} e duas máscaras binárias normalizadas correspondentes, A e B. Então calcula a \textit{\acrshort{HD}} seguindo a \refEq{eq:HD}, e a compara com o limiar \textit{T} definido para o sistema e, caso seja menor, aceita a autenticação e rejeita caso contrário}{fig:metodologia:correspondencia}{width=0.8\textwidth}

\par A etapa de correspondência do \textit{OSIRISv4.1}, assim como a usada pelo autor \cite{daugman2004}, consiste no cálculo da \textit{\acrshort{HD}} usando dois códigos de íris e duas máscaras binárias, mas os métodos utilizados para o seu cálculo são diferentes no \textit{OSIRISv4.1} \cite{osirisv41_doc}. O cálculo consiste em fazer a operação \textit{booleana AND} entre as duas máscaras binárias em posições específicas, definidas pelos pontos de aplicação. Então, a operação \textit{booleana XOR} é feita entre os dois códigos de íris nas posições dos pontos de aplicação que o resultado da operação entre as máscaras é 1, ou seja, não é ruído. Quanto menor o valor resultante, maior a probabilidade dos dois códigos de íris pertencerem à mesma pessoa. Os pontos de aplicação são definidos em um arquivo de configuração e são disponibilizados em \cite{osirisv41} pelos autores. A \refFig{fig:metodologia:processo_matching} ilustra os passos descritos acima para o cálculo da \textit{\acrshort{HD}}.

\par No próximo capítulo, serão descritos os experimentos realizados para validar as métricas \textit{\acrshort{DSMI}} e \textit{\acrshort{FCE}} na arquitetura proposta no diagrama da \refFig{fig:metodologia:arquitetura}.

\figuraBib[H]{img/Metodologia/processo_matching}{Processos realizados para calcular a \textit{\acrfull{HD}} entre os modelos obtidos dos processos de segmentação e codificação de um indivíduo. Operação \textit{booleana AND} é feita entre duas máscaras binárias normalizadas em posições definidas nos pontos de aplicação. A operação \textit{XOR} é então aplicada nos pontos de aplicação que não são ruídos, obtidos no resultado da operação \textit{AND} }{osirisv41_doc}{fig:metodologia:processo_matching}{width=0.8\textwidth}